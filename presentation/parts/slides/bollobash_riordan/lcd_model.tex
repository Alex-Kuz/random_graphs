\frametitle{LCD-модель}


\only<1>{
    Введем такой объект, который называется \textit{линейной хордовой диаграммой}.
}

\only<2-5>{
    Зафиксируем на оси абсцисс на плоскости $2n$ точек: $1, 2, 3, \dots , 2n$.
}

\only<3-5>{
    Разобьем эти точки на пары, и элементы каждой пары соединим дугой, лежащей в верхней полуплоскости.\\
    Полученный объект назовем линейной хордовой диаграммой (linearized chord diagram).\\
    Дуги в нем не могут иметь общих вершин. Количество различных LCD:
} 

\only<4-5>{
    \[
        l_n = \frac{(2n)!}{n!2^n}
    \]
}

\only<6-8>{
    По каждой LCD построим граф с $n$ вершинами и $n$ ребрами. 
}

\only<7-8>{
    Идем слева направо по оси абсцисс, пока не встретим впервые правый конец какой-либо дуги.\\
    Пусть этот конец имеет номер $i_1$. Объявляем набор $\left\{1, . . . , i_1\right\}$ первой вершиной будущего графа.
}

\only<8-8>{
    Снова идем от $i_1+1$ направо до первого правого конца $i_2$ какой-либо дуги.\\
    Объявляем второй вершиной графа набор $\left\{i_1+1, . . . , i_2\right\}$.\\
    И так далее...
}

\only<9-12>{
    Ребра порождаем дугами. То есть две вершины соединяем ребром, если между соответствующими наборами есть дуга.     
}

\only<10-12>{
    Теперь считаем LCD случайной, полагаем вероятность каждой диаграммы равной $1/l_n$.\\
    Возникают случайные графы.    
}

\only<11-12>{
    Можно показать, что такие графы по своим вероятностным характеристикам практически неотличимы от графов $G_1^n$.\\
    Графы с n вершинами и kn ребрами получаем тем же способом, что и в предыдущем пункте.
}
