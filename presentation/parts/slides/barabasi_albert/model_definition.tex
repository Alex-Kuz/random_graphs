\frametitle{Модель БА} 


\begin{rdefinition}    
    Модель Барабаши-Альберт (БА) — алгоритм генерации случайных безмасштабных сетей с использованием принципа \
    предпочтительного присоединения. Безмасштабные сети широко распространены в природных сетях (пищевые цепочки) и сетях, \
    созданных человеком (Интернет, всемирная паутина, сети цитирования, некоторые социальные сети).
\end{rdefinition}

\begin{rdefinition}  
    Безмасштабная сеть - граф, в котором степени вершин распределены по степенному закону, \
     то есть доля вершин со степенью $k$ примерно или асимптотически пропорциональна $k^{-\gamma}$.
\end{rdefinition}
