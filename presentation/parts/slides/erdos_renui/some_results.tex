\frametitle{Интересные результаты с моделью Эрдеша-Реньи}

\only<1>{

    Перечислим некоторые результаты исследований случайных графов при $p_{i, j} = p = \frac{1}{2}$.

    \begin{itemize}[<+(1)->] 
        \item Почти нет Эйлеровых графов.
        \item Почти все графы связны.
        \item Почти все графы гамильтоновы.
        \item Почти все графы имеют диаметр 2.
    \end{itemize} 
}


\only<2->{

    Cправедлива следующая теорема Эрдеша – Реньи (более сильное утверждение о связности).

    \begin{rtheorem}
        Рассмотрим модель $G(n, p)$, где $p = \frac{c \ln n}{n}$. 
        \begin{itemize}[<+(1)->]
            \item Если $c > 1$, то почти всегда случайный граф связен.
            \item Если $c < 1$, то почти всегда случайный граф не является связным.
            \item Если $c = 1$, то вероятность того, что случайный граф связен, стремится к $e^{-1}$. 
        \end{itemize}
    \end{rtheorem}
}