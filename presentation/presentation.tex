\documentclass[t]{beamer}

\usetheme{Ilmenau}
\usecolortheme{beaver}

\usepackage{cmap}					% поиск в PDF
\usepackage{mathtext} 				% русские буквы в формулах
\usepackage[T2A]{fontenc}			% кодировка
\usepackage[utf8]{inputenc}			% кодировка исходного текста
\usepackage[english,russian]{babel}	% локализация и переносы
\usepackage{amssymb}

%% Beamer по-русски
\newtheorem{rtheorem}{Теорема}
\newtheorem{rproof}{Доказательство}
\newtheorem{rexample}{Пример}

%%% Дополнительная работа с математикой
\usepackage{amsmath,amsfonts,amssymb,amsthm,mathtools} % AMS
\usepackage{icomma} % "Умная" запятая: $0,2$ --- число, $0, 2$ --- перечисление
\usepackage{listings}

%% Номера формул
%\mathtoolsset{showonlyrefs=true} % Показывать номера только у тех формул, на которые есть \eqref{} в тексте.
%\usepackage{leqno} % Нумерация формул слева

%% Свои команды
\DeclareMathOperator{\sgn}{\mathop{sgn}}

%% Перенос знаков в формулах (по Львовскому)
\newcommand*{\hm}[1]{#1\nobreak\discretionary{}
{\hbox{$\mathsurround=0pt #1$}}{}}

%%% Работа с картинками
\usepackage{graphicx}  % Для вставки рисунков
\graphicspath{{images/}{images2/}}  % папки с картинками
\setlength\fboxsep{3pt} % Отступ рамки \fbox{} от рисунка
\setlength\fboxrule{1pt} % Толщина линий рамки \fbox{}
\usepackage{wrapfig} % Обтекание рисунков текстом

%%% Работа с таблицами
\usepackage{array,tabularx,tabulary,booktabs} % Дополнительная работа с таблицами
\usepackage{longtable}  % Длинные таблицы
\usepackage{multirow} % Слияние строк в таблице

%%% Программирование
\usepackage{etoolbox} % логические операторы

%%% Другие пакеты
\usepackage{lastpage} % Узнать, сколько всего страниц в документе.
\usepackage{soul} % Модификаторы начертания
\usepackage{csquotes} % Еще инструменты для ссылок
%\usepackage[style=authoryear,maxcitenames=2,backend=biber,sorting=nty]{biblatex}
\usepackage{multicol} % Несколько колонок

%%% Картинки
\usepackage{tikz} % Работа с графикой
\usepackage{pgfplots}
\usepackage{pgfplotstable}

\usepackage{array}
\usepackage{longtable}


\title{Модели случайных графов}

\author{А. Кузнецов ИУ8-74}
\date{October $20^{\mathrm{th}}$, 2018}
\institute{Бауманка}


\begin{document}
    
    \frame{\titlepage}
    
    \section{Introduction}
    \subsection{}  
    \subsubsection{}	
    
        
\begin{frame}
    \frametitle{Calculation software for the TMM training process} 
    \hspace{-0.5cm}
    \begin{tabular}{|b{2cm}|b{1.7cm}|b{1.2cm}|b{1.5cm}|b{1.5cm}|b{1cm}|} \hline
        Program & Symbolic calculations & Data export & Graphical output & Operating system* &  Free license \\ \hline
        \multicolumn{6}{|l|}{Proprietary software}  \\ \hline
        Mathcad & $\times$  & $\times$ & \checkmark & 1 & $\times$ \\ \hline
        Mathematica & \checkmark  & \checkmark & \checkmark & 4 & $\times$ \\ \hline
        MATLAB & \checkmark  & \checkmark & $\times$ & 4 & $\times$ \\ \hline
        \multicolumn{6}{|l|}{Free software}   \\ \hline
        Maxima & \checkmark & \checkmark & \checkmark & >5 & \checkmark \\ \hline
        Scilab & \checkmark & $\times$ & $\times$ & 3 & \checkmark \\ \hline
        
    \end{tabular} \\
    \vspace{0.5cm}
    * Amount of operating systems on which the program can work
\end{frame}	
    
    
    

\end{document}
